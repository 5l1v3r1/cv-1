\documentclass[12pt]{article}
\usepackage[left=1in,right=1in,top=0.1in,bottom=1in]{geometry}

\usepackage{graphicx}
\usepackage{url}
\usepackage[utf8]{inputenc}

\title{\textbf{Statement of Skills}}
\author{Leonardo Uieda}
\date{June 2016}

\begin{document}

\maketitle

% Skills or experience suitable for contributing to GMT development

The following are
the skills that
I believe make me qualified
for making significant contributions
to the GMT project:
%

\begin{itemize}
    \item \textbf{10 years of experience with C programming.}
        My experience with C started in an undergraduate course
        and culminated with the development of
        Tesseroids\footnote{\url{http://tesseroids.leouieda.com/}}.
        Throughout, I have learned to develop command-line interfaces,
        use the build tools Make and SCons,
        and compile and distribute cross-platform C applications.

    \item \textbf{8 years of experience with Python.}
        I have experience with most major scientific libraries:
        numpy, scipy, matplotlib, pandas, jupyter, mayavi, etc.
        Through my project Fatiando a
        Terra\footnote{\url{http://www.fatiando.org/}},
        I have learned how to develop,
        test, document, and distribute Python packages,
        as well as implement embarrassingly parallel tasks
        using built-in Python
        modules\footnote{\url{https://github.com/fatiando/fatiando/blob/master/fatiando/gravmag/tesseroid.py}}.
        I have experience using Cython and numba to optimize critical points
        of the Fatiando library, for example a finite-difference elastic wave
        modeling
        code\footnote{\url{https://github.com/fatiando/fatiando/blob/master/fatiando/seismic/_wavefd.pyx}}.

    \item \textbf{Develop automated tests and use continuous-integration
        services.}
        Both Tesseroids and Fatiando a Terra have automated test suites
        that are run for every new commit by the Travis
        CI\footnote{\url{https://travis-ci.org/}} service.

    \item \textbf{Version control systems Subversion, git and
        Mercurial}.
        The initial development of Fatiando a Terra and Tesseroids was managed
        using Subversion. I later migrated both projects to Mercurial and
        finally to git in order to take advantage of the Github hosting
        website. Today, all of my research projects, papers, course material,
        and presentations are stored in git repositories and Fatiando a Terra
        uses the Github Pull Request model for collaborative development.

    \item \textbf{Making interactive documents and rich displays using the
        Jupyter notebook.}
        I use the Jupyter notebook\footnote{\url{http://jupyter.org/}}
        routinely to create interactive documents for my research
        and
        courses\footnote{\url{http://nbviewer.jupyter.org/github/leouieda/geofisica1/tree/master/notebooks/}}.
        I know how to leverage the rich display functionalities of
        Jupyter to embed plots, HTML snippets, and even
        video\footnote{\url{http://nbviewer.jupyter.org/github/fatiando/prototypes/blob/master/wavefd-simulation-class.ipynb}}
        into the notebook documents.
\end{itemize}

\end{document}
