%%%%%%%%%%%%%%%%%%%%%%%%%%%%%%%%%%%%%%%%%%%%%%%%%%%%%%%%%%%%%%%%%%%%%%%%%%%%%%%
% A clean template for an academic CV
%
% Uses tabularx to create two column entries (date and job/edu/citation).
% Defines commands to make adding entries simpler.
%
%%%%%%%%%%%%%%%%%%%%%%%%%%%%%%%%%%%%%%%%%%%%%%%%%%%%%%%%%%%%%%%%%%%%%%%%%%%%%%%

\documentclass[11pt, a4paper]{article}

% Identifying information
\newcommand{\Title}{Curriculum Vit\ae}
\newcommand{\FirstName}{Leonardo}
\newcommand{\LastName}{Uieda}
\newcommand{\Initials}{L}
\newcommand{\MyName}{\FirstName\ \LastName}
\newcommand{\Me}{\textbf{\LastName, \Initials}}  % For citations
\newcommand{\Email}{leouieda@gmail.com}
\newcommand{\Website}{www.leouieda.com}
\newcommand{\ORCID}{0000-0001-6123-9515}
\newcommand{\Affiliation}{
    Department of Earth Sciences
    -- SOEST --
    University of Hawai'i at M\={a}noa
}
\newcommand{\Address}{
    POST Building Suite 821, 1680 East-West Rd, Honolulu, HI, USA, 96822
}

% Names for citing coauthors
\newcommand{\Val}{Barbosa, VCF}
\newcommand{\Bi}{Oliveira Jr, VC}
\newcommand{\Paul}{Wessel, P}
\newcommand{\Bridget}{Smith-Konter, B}
\newcommand{\Eric}{Xu, X}
\newcommand{\David}{Sandwell, DT}
\newcommand{\Carla}{Braitenberg, C}
\newcommand{\JB}{Silva, JBC}
\newcommand{\Dai}{Sales, DP}
\newcommand{\Figura}{Melo, FF}
\newcommand{\Dio}{Carlos, DU}
\newcommand{\BragaVale}{Braga, MA}
\newcommand{\YLi}{Li, Y}
\newcommand{\Angeli}{Angeli, G}
\newcommand{\Peres}{Peres, G}
\newcommand{\Everton}{Bomfim, EP}
\newcommand{\Eder}{Molina, E}
\newcommand{\Gomes}{Gomes, AAS}
\newcommand{\Santiago}{Soler, SR}
\newcommand{\Agustina}{Pesce, A}
\newcommand{\Gimenez}{Gimenez, ME}
\newcommand{\Kristoffer}{Hallam, KAT}
\newcommand{\Guangdong}{Zhao, G}
\newcommand{\Bo}{Chen, B}
\newcommand{\JLiu}{Liu, J}
\newcommand{\LChen}{Chen, L}
\newcommand{\RGuo}{Guo, R}


% Template configuration
%%%%%%%%%%%%%%%%%%%%%%%%%%%%%%%%%%%%%%%%%%%%%%%%%%%%%%%%%%%%%%%%%%%%%%%%%%%%%%%

% Control the font size
\usepackage{anyfontsize}

% Template variables for styling
\newcommand{\TablePad}{\vspace{-0.4cm}}
\newcommand{\SoftwareTitle}[1]{{\bfseries #1}}
\newcommand{\TableTitle}[1]{{\fontsize{12pt}{0}\selectfont \itshape #1}}
\newcommand{\Invited}{[\textbf{Invited}] -- }

% For fancy and multipage tables
\usepackage{tabularx}
\usepackage{ltablex}

% Define a new environment to place all CV entries in a 2-column table.
% Left column are the dates, right column the entries.
\usepackage{environ}
\NewEnviron{EntriesTable}{
    \TablePad
    \begin{tabularx}{\textwidth}{@{}p{0.10\textwidth} p{0.9\textwidth}@{}}
        \BODY
    \end{tabularx}
}

% Macros to add links
\newcommand{\DOI}[1]{doi:\href{https://doi.org/#1}{#1}}
\newcommand{\Youtube}[1]{[recording: \href{https://youtu.be/#1}{youtu.be/#1}]}
% Macros to set the year and duration on the left column
\newcommand{\Duration}[2]{\fontsize{10pt}{0}\selectfont #1--#2}
\newcommand{\Year}[1]{\fontsize{10pt}{0}\selectfont #1}

% Define command to insert month name and year as date
\usepackage{datetime}
\newdateformat{monthyear}{\monthname[\THEMONTH], \THEYEAR}

% Set the page margins
\usepackage[left=1in,right=1in,top=1in,bottom=1in]{geometry}

% No indentation
\setlength\parindent{0cm}

% Increase the line spacing
\renewcommand{\baselinestretch}{1.1}
\renewcommand{\arraystretch}{1.5}

% Remove space between items in itemize and enumerate
\usepackage{enumitem}
\setlist{nosep}

% Use custom colors
\usepackage[usenames,dvipsnames]{xcolor}

% Set fonts. Requires compilation with xelatex
%\usepackage{fontspec}
%\setmainfont[BoldFont=SourceSansPro-Semibold]{SourceSansPro-Light}
%\setmonofont{Source Code Pro}

% Configure the font style for sections
\usepackage{sectsty}
\sectionfont{\vspace{0.5cm}\bfseries\fontsize{12pt}{0}\selectfont\uppercase}
\subsectionfont{\vspace{0.2cm}\mdseries\fontsize{12pt}{0}\selectfont\uppercase}

% Set fancy headers
\usepackage{fancyhdr}
\pagestyle{fancy}
\fancyhf{}
\chead{
    \itshape
    \fontsize{10pt}{12pt}\selectfont
    \MyName
    \hspace{0.2cm} -- \hspace{0.2cm}
    \Title
    \hspace{0.2cm} -- \hspace{0.2cm}
    \monthyear\today
}
\rhead{}
\cfoot{\fontsize{10pt}{0}\selectfont \thepage}
\renewcommand{\headrulewidth}{0pt}

% Metadata for the PDF output and control of hyperlinks
\usepackage[colorlinks=true]{hyperref}
\hypersetup{
    pdftitle={\MyName\ - \Title},
    pdfauthor={\MyName},
    linkcolor=blue,
    citecolor=blue,
    filecolor=black,
    urlcolor=MidnightBlue
}
%%%%%%%%%%%%%%%%%%%%%%%%%%%%%%%%%%%%%%%%%%%%%%%%%%%%%%%%%%%%%%%%%%%%%%%%%%%%%%%



\begin{document}

% No header for the first page
\thispagestyle{empty}

%%%%%%%%%%%%%%%%%%%%%%%%%%%%%%%%%%%%%%%%%%%%%%%%%%%%%%%%%%%%%%%%%%%%%%%%%%%%%%%
% HEADER
\begin{center}
    {\fontsize{24pt}{0}\selectfont \MyName}
    \\[0.5cm]
    {\fontsize{10pt}{0}\selectfont \Title \, | \, \monthyear\today}
    \\[0.5cm]
    {\fontsize{10pt}{0}\selectfont
        \Affiliation
        \\[0.2cm]
        \Address
        \\[0.08cm]
        ORCID: \href{http://orcid.org/\ORCID}{\ORCID}
        \, | \,
        \href{mailto:\Email}{\Email}
        \, | \,
        \href{http://\Website}{\Website}
    }
\end{center}


%%%%%%%%%%%%%%%%%%%%%%%%%%%%%%%%%%%%%%%%%%%%%%%%%%%%%%%%%%%%%%%%%%%%%%%%%%%%%%%
\section*{Professional Appointments}

\begin{EntriesTable}
    \Duration{2017}{}  &
    \textbf{Visiting Research Scholar},
    Department of Earth Sciences,
    School of Ocean and Earth Science and Technology,
    University of Hawai'i at M\={a}noa, USA
    \\
    \Duration{2014}{2018}  &
    \textbf{Assistant Professor},
    Departamento de Geologia Aplicada, Faculdade de Geologia,
    Universidade do Estado do Rio de Janeiro, Brazil
\end{EntriesTable}


%%%%%%%%%%%%%%%%%%%%%%%%%%%%%%%%%%%%%%%%%%%%%%%%%%%%%%%%%%%%%%%%%%%%%%%%%%%%%%%
\section*{Education}

\begin{EntriesTable}
    \Year{2016}  &
    \textbf{PhD in Geophysics}, Observatório Nacional, Brazil
    \newline
    Thesis: Forward modeling and inversion of gravitational fields in
    spherical coordinates
    \\
    \Year{2011}  &
    \textbf{MSc in Geophysics}, Observatório Nacional, Brazil
    \\
    \Year{2009}  &
    \textbf{BSc in Geophysics}, Universidade de São Paulo, Brazil
    \\
    \Year{2008}  &
    \textbf{International Exchange}, York University, Canada
\end{EntriesTable}


%%%%%%%%%%%%%%%%%%%%%%%%%%%%%%%%%%%%%%%%%%%%%%%%%%%%%%%%%%%%%%%%%%%%%%%%%%%%%%%
\section*{Awards \& Honors}

\begin{EntriesTable}
    \Year{2017}  &
    Brazilian Geophysical Society (SBGf) Award for \textbf{Best PhD Thesis}
    of 2015 -- 2017
    \\
    \Year{2016}  &
    Universidade do Estado do Rio de Janeiro, Brazil, School of Geology
    \textbf{Teaching Award} given by the graduating class of 2016
    \\
    \Duration{2014}{2018}  &
    QUALITEC/UERJ Grant for training a technician for the Laboratory of
    Exploration Geophysics - Universidade do Estado do Rio de Janeiro
    \\
    \Duration{2011}{2015}  &
    Brazilian Ministry of Education CAPES \textbf{PhD Research Scholarship}
    \\
    \Year{2011}  &
    SEG Near Surface Geophysics Section \textbf{Student Travel Grant} to
    present at the SEG Annual Meeting, San Antornio, TX, USA
    \\
    \Year{2011}  &
    EAGE \textbf{PACE Student Travel Grant} to present at the 73rd EAGE
    Conference \& Exhibition, Vienna, Austria
    \\
    \Duration{2010}{2011}  &
    Brazilian Ministry of Education CAPES \textbf{Masters Research Scholarship}
    \\
    \Year{2008}  &
    Brazilian Geophysical Society (SBGf) \textbf{Undergraduate Research
    Scholarship}
    \\
    \Year{2005}  &
    São Paulo Research Foundation (FAPESP) \textbf{Undergraduate Research
    Scholarship}
\end{EntriesTable}


%%%%%%%%%%%%%%%%%%%%%%%%%%%%%%%%%%%%%%%%%%%%%%%%%%%%%%%%%%%%%%%%%%%%%%%%%%%%%%%
\section*{Grants \& Fellowships}

\begin{EntriesTable}
    \Duration{2018}{2020}  &
    NSF-EAR: ``The EarthScope/GMT Analysis and Visualization Toolbox''.
    PI: \Paul, \textbf{co-PI}: \Me, co-PI: \Bridget.
    University of Hawai'i at M\={a}noa.
    Award number: 1829371.
    \$174,975.
\end{EntriesTable}


%%%%%%%%%%%%%%%%%%%%%%%%%%%%%%%%%%%%%%%%%%%%%%%%%%%%%%%%%%%%%%%%%%%%%%%%%%%%%%%
\section*{Publications}

Source code, data, and PDFs for most articles are available
at \href{http://www.leouieda.com/papers}{leouieda.com/papers}

\subsection*{Peer-Reviewed}

\begin{EntriesTable}
\Year{in review}  &
    \Santiago, \Agustina, \Me, \Gimenez.
    Gravitational field calculation in spherical coordinates using variable densities in
    depth.
    In review at \emph{Geophysical Journal International}.
    \\
    ~ &
    \Guangdong, \Bo, \Me, \JLiu, \LChen, \RGuo.
    Efficient 3D large-scale forward-modeling and inversion of gravitational fields in
    spherical coordinates with application to lunar mascons.
    In review at \emph{JGR Solid Earth}.
    \\
    ~ &
    \Bi, \Me, \Kristoffer, \Val.
    Should geophysicists use the gravity disturbance or the anomaly?
    In review at \emph{Geophysics}.
    Preprint:
    \href{https://www.leouieda.com/papers/use-the-disturbance.html}{leouieda.com/papers/use-the-disturbance.html}
    \\
\Year{2018}  &
    \Me. Verde: Processing and gridding spatial data using Green's functions.
    \emph{Journal of Open Source Software}.
    \DOI{10.21105/joss.00957}.
    \\
\Year{2017}  &
    \Me, \Val.
    Fast non-linear gravity inversion in spherical coordinates with application
    to the South American Moho,
    \emph{Geophysical Journal International},
    \DOI{10.1093/gji/ggw390}.
    \\
\Year{2016}  &
    \Me, \Val, \Carla.
    Tesseroids: forward modeling gravitational fields in spherical coordinates,
    \emph{Geophysics},
    \DOI{10.1190/geo2015-0204.1}.
    \\
    ~ &
    \Dio, \Me, \Val.
    How two gravity-gradient inversion methods can be used to reveal different
    geologic features of ore deposit - A case study from the Quadrilátero
    Ferrífero (Brazil),
    \emph{Journal of Applied Geophysics},
    \DOI{10.1016/j.jappgeo.2016.04.011}.
    \\
\Year{2015}  &
    \Bi, \Dai, \Val, \Me.
    Estimation of the total magnetization direction of approximately spherical
    bodies,
    \emph{Nonlinear Processes in Geophysics},
    \DOI{10.5194/npg-22-215-2015}.
    \\
\Year{2014}  &
    \Dio, \Me, \Val.
    Imaging iron ore from the Quadrilátero Ferrífero (Brazil) using geophysical
    inversion and drill hole data,
    \emph{Ore Geology Reviews},
    \DOI{10.1016/j.oregeorev.2014.02.011}.
    \\
\Year{2013}  &
    \Figura, \Val, \Me, \Bi, \JB.
    Estimating the nature and the horizontal and vertical positions of 3D
    magnetic sources using Euler deconvolution,
    \emph{Geophysics},
    \DOI{10.1190/geo2012-0515.1}.
    \\
    ~ &
    \Bi, \Val, \Me.
    Polynomial equivalent layer,
    \emph{Geophysics},
    \DOI{10.1190/geo2012-0196.1}.
    \\
\Year{2012}  &
    \Me, \Val.
    Robust 3D gravity gradient inversion by planting anomalous densities,
    \emph{Geophysics},
    \DOI{10.1190/geo2011-0388.1}.
\end{EntriesTable}


\subsection*{Peer-Reviewed Conference proceedings}

\begin{EntriesTable}
\Year{2014}  &
    \Figura, \Val, \Me, \Bi, \JB.
    A Single Euler Solution Per Anomaly,
    \emph{76th EAGE Conference and Exhibition 2014},
    \DOI{10.3997/2214-4609.20140891}.
    \\
\Year{2013}  &
    \Me, \Bi, \Val.
    Modeling the Earth with Fatiando a Terra,
    \emph{Proceedings of the 12th Python in Science Conference}.
    \\
\Year{2012}  &
    \Me, \Val.
    Use of the ``shape-of-anomaly'' data misfit in 3D inversion by planting
    anomalous densities,
    \emph{SEG Technical Program Expanded Abstracts},
    \DOI{10.1190/segam2012-0383.1}.
    \\
    ~ &
    \Dio, \Me, \YLi, \Val, \BragaVale, \Angeli, \Peres.
    Iron ore interpretation using gravity-gradient inversions in the Carajás, Brazil.
    \emph{SEG Technical Program Expanded Abstracts},
    \DOI{10.1190/segam2012-0525.1}.
    \\
\Year{2011}  &
    \Me, \Everton, \Carla, \Eder.
    Optimal forward calculation method of the Marussi tensor due to a geologic
    structure at GOCE height,
    \emph{Proceedings of the 4th International GOCE User Workshop}.
    \\
    ~ &
    \Me, \Val.
    Robust 3D gravity gradient inversion by planting anomalous densities,
    \emph{SEG Technical Program Expanded Abstracts},
    \DOI{10.1190/1.3628201}.
    \\
    ~ &
    \Me, \Val.
    3D gravity inversion by planting anomalous densities.
    \emph{12th International Congress of the Brazilian Geophysical Society},
    \DOI{10.1190/sbgf2011-179}.
    \\
    ~ &
    \Me, \Val.
    3D gravity gradient inversion by planting density anomalies.
    \emph{73th EAGE Conference and Exhibition incorporating SPE EUROPEC},
    \DOI{10.3997/2214-4609.20149567}.
    \\
    ~ &
    \Dio, \Me, \Val, \BragaVale, \Gomes.
    In-depth imaging of an iron orebody from Quadrilatero Ferrifero using 3D
    gravity gradient inversion,
    \emph{SEG Technical Program Expanded Abstracts},
    \DOI{10.1190/1.3628219}.
    \\
    ~ &
    \Dio, \Val, \Me, \BragaVale.
    Inversão de Dados de Aerogradiometria Gravimétrica 3D-FTG Aplicada a
    Exploração Mineral na Região do Quadrilátero Ferrífero,
    \emph{12th International Congress of the Brazilian Geophysical Society},
    \DOI{10.1190/sbgf2011-243}.
\end{EntriesTable}


\subsection*{Other publications}

\begin{EntriesTable}
\Year{2017}  &
    \Me.
    Step-by-step NMO correction,
    \emph{The Leading Edge},
    \DOI{10.1190/tle36020179.1}.
    \\
\Year{2014}  &
    \Me, \Bi, \Val.
    Geophysical tutorial: Euler deconvolution of potential-field data,
    \emph{The Leading Edge},
    \DOI{10.1190/tle33040448.1}.
\end{EntriesTable}


\subsection*{Open Datasets}

\begin{EntriesTable}
\Year{2017}  &
    \Me, \Val.
    A gravity-derived Moho model for South America: source code, data, and
    model results from ``Fast non-linear gravity inversion in spherical
    coordinates with application to the South American Moho''.
    \DOI{10.6084/m9.figshare.3987267}
\end{EntriesTable}


\subsection*{Open-source Software}

I work on several open-source projects, all of which are available through
my Github profile
\href{https://github.com/leouieda/}{github.com/leouieda}.

\begin{EntriesTable}
    \Duration{2017}{} &
    \textbf{GMT/Python}
    --
    \href{http://www.gmtpython.xyz}{www.gmtpython.xyz}
    --
    A Python interface for the Generic Mapping Tools.
    \\
    \Duration{2010}{} &
    \textbf{Fatiando a Terra}
    --
    \href{http://www.fatiando.org}{www.fatiando.org}
    --
    Toolkit for geophysical data analysis, forward modeling, and inversion.
    Language: Python.
    \\
    \Duration{2009}{2016} &
    \textbf{Tesseroids}
    --
    \href{http://www.tesseroids.org}{www.tesseroids.org}
    --
    Forward modeling of gravitational fields in spherical coordinates.
    Language: C.
\end{EntriesTable}


%%%%%%%%%%%%%%%%%%%%%%%%%%%%%%%%%%%%%%%%%%%%%%%%%%%%%%%%%%%%%%%%%%%%%%%%%%%%%%%
\section*{Teaching}

All educational material developed for these courses is available at
\href{http://www.leouieda.com/teaching}{leouieda.com/teaching}

\subsection*{Undergraduate -- Universidade do Estado do Rio de Janeiro}

\begin{EntriesTable}
    \Duration{2014}{2016}  &
    Special Mathematics I: Introduction to Programming and Numerical Analysis
    \\
    \Duration{2014}{2016}  &
    Geophysics I: Gravity and magnetic methods
    \\
    \Duration{2014}{2016}  &
    Geophysics II: Exploration Seismology
    \\
    \Year{2015}  &
    Introduction to Geology
\end{EntriesTable}


\subsection*{Workshops and Short Courses}

\begin{EntriesTable}
\Year{future}  &
    Best Practices for Modern Open-Source Research Codes (half day workshop)
    \newline
    AGU 2018, Washington DC, USA.
    \\
\Year{2017}  &
    Introduction to Python (6 hour workshop)
    \newline
    Department of Geology and Geophysics -- University of Hawai'i at M\={a}noa,
    USA
    \\
\Year{2016}  &
    Python for Geologists (6 hour workshop)
    \newline
    School of Geology -- Universidade do Estado do Rio de Janeiro, Brazil
    \\
    ~  &
    Python for Earth Scientists (30 hour short course)
    \newline
    Department of Geophysics -- Universidade de São Paulo, Brazil
    \\
\Year{2014}  &
    Introduction to Geophysical Inversion (16 hour short course)
    \newline
    Institute of Geosciences -- Universidade de Brasília, Brazil
    \\
\Year{2011}  &
    Introduction to Geophysical Inversion (30 hour short course)
    \newline
    Department of Geophysics -- Universidade de São Paulo, Brazil
\end{EntriesTable}


%%%%%%%%%%%%%%%%%%%%%%%%%%%%%%%%%%%%%%%%%%%%%%%%%%%%%%%%%%%%%%%%%%%%%%%%%%%%%%%
\section*{Presentations}

Slides, posters, and abstracts for all presentations are available at
\href{http://www.leouieda.com/talks/}{leouieda.com/talks}
and
\\
\href{http://www.leouieda.com/posters/}{leouieda.com/posters}
\\
\begin{EntriesTable}
\Year{Future}  &
    \Me, \Eric, \Paul, \David.
    Coupled Interpolation of Three-component GPS Velocities,
    \emph{AGU 2018},
    Washington DC, USA.
    \\
\Year{2018}  &
    \Me, \Paul.
    Building an object-oriented Python interface for the Generic Mapping Tools,
    \emph{Scipy 2018},
    Austin, USA.
    \Youtube{6wMtfZXfTRM}
    \\
    ~ &
    \Me, \David, \Paul.
    Joint Interpolation of 3-component GPS Velocities Constrained by
    Elasticity,
    \emph{AOGS $15^{th}$ Annual Meeting},
    Honolulu, USA.
    \\
    ~ &
    \Me, \Paul.
    Integrating the Generic Mapping Tools with the Scientific Python Ecosystem,
    \emph{AOGS $15^{th}$ Annual Meeting},
    Honolulu, USA.
    \\
\Year{2017}  &
    \Invited{}
    \Me, \Paul.
    Nurturing reliable and robust open-source scientific software,
    \emph{AGU Fall Meeting 2017},
    New Orleans, USA.
    \Youtube{0GO4ZZ5Ry6M}
    \\
    ~  &
    \Me, \Paul.
    A modern Python interface for the Generic Mapping Tools,
    \emph{AGU Fall Meeting 2017},
    New Orleans, USA.
    \\
    ~  &
    \Me, \Paul.
    Bringing the Generic Mapping Tools to Python,
    \emph{Scipy 2017},
    Austin, USA.
    \Youtube{93M4How7R24}
    \\
    ~ &
    \Me.
    Inverting gravity to map the Moho: A new method and the open source
    software that made it possible,
    \emph{Department of Geology and Geophysics, University of Hawai'i at
    M\={a}noa},
    Honolulu, USA.
    \\
\Year{2016}  &
    \Invited{}
    \Me.
    Fatiando a Terra: construindo uma base para ensino e pesquisa de geofísica,
    \emph{Observatório Nacional},
    Rio de Janeiro, Brazil.
    \\
\Year{2015}  &
    \Invited{}
    \Me.
    Fatiando a Terra: construindo uma base para ensino e pesquisa de geofísica,
    \emph{Universidade de São Paulo},
    São Paulo, Brazil.
    \\
\Year{2014}  &
    \Me, et al.
    Using Fatiando a Terra to solve inverse problems in geophysics,
    \emph{Scipy 2014},
    Austin, USA.
    \\
    ~ &
    \Me, et al.
    Gravity inversion in spherical coordinates using tesseroids,
    \emph{EGU General Assembly 2014},
    Vienna, Austria.
    \\
\Year{2013}  &
    \Me, et al.
    Modeling the Earth with Fatiando a Terra,
    \emph{Scipy 2013},
    Austin, USA.
    \Youtube{Ec38h1oB8cc}
    \\
    ~ &
    \Me, et al.
    3D magnetic inversion by planting anomalous densities,
    \emph{AGU Meeting of the Americas},
    Cancun, Mexico.
    \\
\Year{2012}  &
    \Dio, \Me, et al.
    Iron ore interpretation using gravity-gradient inversions in the Carajás,
    Brazil,
    \emph{SEG Annual Meeting 2012},
    Las Vegas, USA.
    \\
    ~ &
    \Me, et al.
    Use of the ``shape-of-anomaly'' data misfit in 3D inversion by planting
    anomalous densities,
    \emph{SEG Annual Meeting 2012},
    Las Vegas, USA.
    \\
    ~ &
    \Me, et al.
    Rapid 3D inversion of gravity and gravity gradient data to test geologic
    hypotheses,
    \emph{International Symposium on Gravity, Geoid and Height Systems},
    Venice, Italy.
    \\
\Year{2011}  &
    \Me, et al.
    Robust 3D gravity gradient inversion by planting anomalous densities,
    \emph{SEG Annual Meeting 2011},
    San Antonio, USA.
    \\
    ~ &
    \Me, et al.
    3D gravity inversion by planting anomalous densities,
    \emph{Internation Congress of the Brazilian Geophysical Society},
    Rio de Janeiro, Brazil.
    \\
    ~ &
    \Me, et al.
    Optimal forward calculation method of the Marussi tensor due to a geologic
    structure at GOCE height,
    \emph{4th International GOCE User Workshop},
    Munich, Germany.
    \\
    ~ &
    \Me, et al.
    3D gravity gradient inversion by planting density anomalies,
    \emph{73th EAGE Conference and Exhibition incorporating SPE EUROPEC},
    Vienna, Austria.
    \\
\Year{2010}  &
    \Me, et al.
    Computation of the gravity gradient tensor due to topographic masses using
    tesseroids,
    \emph{AGU Meeting of the Americas},
    Foz do Iguaçu, Brazil.
    \\
\Year{2008}  &
    \Me, et al.
    Utilização de tesseróides na modelagem de dados de gradiometria
    gravimétrica,
    \emph{XIII Simpósio de Iniciação Científica do IAG-USP},
    São Paulo, Brazil.
    \\
\Year{2006}  &
    \Me, et al.
    Paleomagnetismo e mineralogia magnética dos diques cambrianos de Maravilhas
    e Prata (PB),
    \emph{XI Simpósio de Iniciação Científica do IAG/USP},
    São Paulo, Brazil.
\end{EntriesTable}


%%%%%%%%%%%%%%%%%%%%%%%%%%%%%%%%%%%%%%%%%%%%%%%%%%%%%%%%%%%%%%%%%%%%%%%%%%%%%%%
\section*{Academic Service \& Affiliations}

\subsection*{Committees}

\begin{EntriesTable}
    \Year{2015} & Chairman of the Election Committee for the deans of the
    University and the School of Geology, Universidade do Estado do Rio de
    Janeiro
\end{EntriesTable}


\subsection*{Reviewer}

Geophysical Journal International
--
Journal of Geodesy
--
Pure and Applied Geophysics
--
Journal of Applied Geophysics
--
Geophysical Prospecting
--
Geophysics
--
Central European Journal of Geosciences
--
Computers \& Geosciences
--
Journal of Open Source Software


\subsection*{Affiliations}

American Geophysical Union
--
Society of Exploration Geophysicists
--
Geological Society of America


%%%%%%%%%%%%%%%%%%%%%%%%%%%%%%%%%%%%%%%%%%%%%%%%%%%%%%%%%%%%%%%%%%%%%%%%%%%%%%%
\section*{Languages}

\TablePad
\begin{tabularx}{\textwidth}{@{}p{0.15\textwidth} p{0.85\textwidth}@{}}
    Portuguese & Native
    \\
    English & Fluent (TOEFL iBT score 115/120)
\end{tabularx}

\end{document}
