\documentclass[11pt, a4paper]{article}


\newcommand{\FirstName}{Leonardo}
\newcommand{\LastName}{Uieda}
\newcommand{\MyName}{\FirstName\ \LastName}
\newcommand{\Title}{Curriculum Vit\ae}
\newcommand{\Email}{leouieda@gmail.com}
\newcommand{\ORCID}{0000-0001-6123-9515}
\newcommand{\TablePad}{\vspace{-0.4cm}}
%\newcommand{\Item}{$\bullet$ \hspace{0.1cm}}
\newcommand{\Item}{}
\newcommand{\SoftwareTitle}[1]{{\fontsize{13pt}{0}\selectfont \bfseries #1}}
\newcommand{\TableTitle}[1]{{\fontsize{14pt}{0}\selectfont \itshape #1}}
% Macros to add links to code and PDF versions of articles
\newcommand{\Code}[1]{[\href{#1}{code}]}
\newcommand{\PDF}[1]{[\href{#1}{pdf}]}
\newcommand{\Slides}[1]{[\href{#1}{slides}]}
\newcommand{\Poster}[1]{[\href{#1}{poster}]}
\newcommand{\DOI}[1]{doi:\href{https://doi.org/#1}{#1}}
% Macros to set the year and duration
\newcommand{\Duration}[2]{\fontsize{10pt}{0}\selectfont #1\ --\ #2}
\newcommand{\Year}[1]{\fontsize{10pt}{0}\selectfont #1}


\usepackage{graphicx}
\usepackage{tabularx}
% For multipage tables
\usepackage{ltablex}

% Define command to insert month name and year as date
\usepackage{datetime}
\newdateformat{monthyear}{\monthname[\THEMONTH], \THEYEAR}

% Set the page margins
\usepackage[left=1in,right=1in,top=1in,bottom=1in]{geometry}

% No indentation
\setlength\parindent{0cm}

% Increase the line spacing
\renewcommand{\baselinestretch}{1.1}
\renewcommand{\arraystretch}{1.5}

% Remove space between items in itemize and enumerate
\usepackage{enumitem}
\setlist{nosep}

% Use custom colors
\usepackage[usenames,dvipsnames]{xcolor}

% Set fonts. Requires compilation with xelatex
\usepackage{fontspec}
\setmainfont[BoldFont=SourceSansPro-Semibold]{SourceSansPro-Light}
\setmonofont{Source Code Pro}
% Configure the font style for sections
\usepackage{sectsty}
\sectionfont{\vspace{0.2cm}\mdseries\fontsize{16pt}{0}\selectfont\uppercase}
\subsectionfont{\itshape\mdseries\fontsize{14pt}{0}\selectfont}
% Control the font size
\usepackage{anyfontsize}

% Set headers
\usepackage{fancyhdr}
\pagestyle{fancy}
\fancyhf{}
\chead{
    \itshape
    \fontsize{10pt}{12pt}\selectfont
    \MyName
    \hspace{0.2cm} - \hspace{0.2cm}
    \Title
    \hspace{0.2cm} - \hspace{0.2cm}
    \monthyear\today
}
\rhead{}
\cfoot{\fontsize{10pt}{0}\selectfont \thepage}
\renewcommand{\headrulewidth}{0pt}

% Metadata for the PDF output and control of hyperlinks
\usepackage[colorlinks=true]{hyperref}
\hypersetup{
    pdftitle={\MyName\ - \Title},
    pdfauthor={\MyName},
    linkcolor=blue,
    citecolor=blue,
    filecolor=black,
    urlcolor=MidnightBlue
}



\begin{document}

% No header for the first page
\thispagestyle{empty}

%%%%%%%%%%%%%%%%%%%%%%%%%%%%%%%%%%%%%%%%%%%%%%%%%%%%%%%%%%%%%%%%%%%%%%%%%%%%%%%
% HEADER
\begin{tabular}{@{}l c@{}}
    \parbox{0.6\textwidth}{
        {\fontsize{36pt}{0}\selectfont \MyName}
        \\[0.5cm]
        {\fontsize{13pt}{0}\selectfont \Title \, | \, \monthyear\today}
    } &
    \parbox{0.4\textwidth}{
        \fontsize{10pt}{12pt}\selectfont
        Department of Geology and Geophysics, SOEST
        \\
        University of Hawai'i at Manoa
        \\
        1680 East-West Rd, POST 804
        \\
        96822 Honolulu, HI, USA
        \\
        ORCID: \href{http://orcid.org/\ORCID}{\ORCID}
        \\
        \href{mailto:\Email}{\Email}
        \, | \,
        \href{http://www.leouieda.com}{www.leouieda.com}
    }
\end{tabular}

\vspace{0.5cm}


%%%%%%%%%%%%%%%%%%%%%%%%%%%%%%%%%%%%%%%%%%%%%%%%%%%%%%%%%%%%%%%%%%%%%%%%%%%%%%%
\section*{Research interests}

\TablePad
\begin{tabularx}{\textwidth}{@{}X X X@{}}
    \Item Gravity and magnetics
    &
    \Item Open-source scientific software
    &
    ~
    \\
    \Item Inverse problem theory
    &
    \Item Reproducibility of computations
    &
    ~
\end{tabularx}


%%%%%%%%%%%%%%%%%%%%%%%%%%%%%%%%%%%%%%%%%%%%%%%%%%%%%%%%%%%%%%%%%%%%%%%%%%%%%%%
\section*{Professional experience}

\TablePad
\begin{tabularx}{\textwidth}{@{}l X}
    \Duration{02/2017}{present}  &
    \textbf{Visiting Research Scholar} (postdoc),
    University of Hawaii at Manoa, USA
    \newline
    \textit{Working with Prof. Paul Wessel to build a Python interface for
    the Generic Mapping Tools}
    \\
    \Duration{02/2014}{present}  &
    \textbf{Assistant Professor},
    Universidade do Estado do Rio de Janeiro, Brazil
    \newline
    \textit{Professor of Geophysics at the Department of Applied Geology.
    Responsible for the Laboratory of Exploration Geophysics (LAGEX).}
    \\
    \Duration{02/2011}{03/2011}  &
    \textbf{Visiting Researcher},
    University of Trieste, Italy
    \newline
    \textit{Working with professor Carla Braitenberg to develop version 1.0 of
    the open-source software Tesseroids.}
\end{tabularx}


%%%%%%%%%%%%%%%%%%%%%%%%%%%%%%%%%%%%%%%%%%%%%%%%%%%%%%%%%%%%%%%%%%%%%%%%%%%%%%%
\section*{Education}

\TablePad
\begin{tabularx}{\textwidth}{@{}l X}
    \Duration{11/2011}{04/2016}  &
    \textbf{PhD in Geophysics}, Observatório Nacional, Brazil
    \newline
    Thesis: \textit{Forward modeling and inversion of gravitational fields in
    spherical coordinates}
    %\newline
    \Code{https://github.com/leouieda/phd-thesis}
    \PDF{http://www.leouieda.com/about/phd.html}
    \\
    \Duration{03/2010}{10/2011}  &
    \textbf{MSc in Geophysics}, Observatório Nacional, Brazil
    \newline
    Thesis: \textit{Robust 3D gravity gradient inversion by planting anomalous
    densities}
    %\newline
    \Code{https://github.com/pinga-lab/paper-planting-densities}
    \PDF{http://www.leouieda.com/about/masters.html}
    \\
    \Duration{08/2008}{05/2009}  &
    \textbf{International Exchange}, York University, Canada
    \\
    \Duration{03/2004}{12/2009}  &
    \textbf{BSc in Geophysics}, Universidade de São Paulo, Brazil
    \newline
    Thesis: \textit{Forward modeling of the gravity gradient tensor using
    tesseroids}
    %\newline
    \Code{https://github.com/leouieda/barchelor-thesis}
    \PDF{http://www.leouieda.com/about/bachelors.html}
\end{tabularx}


%%%%%%%%%%%%%%%%%%%%%%%%%%%%%%%%%%%%%%%%%%%%%%%%%%%%%%%%%%%%%%%%%%%%%%%%%%%%%%%
\section*{Funding}

\TablePad
\begin{tabularx}{\textwidth}{@{}l X}
    \Duration{11/2014}{11/2017}  &
    QUALITEC/UERJ program for training a technician for the Laboratory of
    Exploration Geophysics
    \textbf{(BRL \$154,800 as a scholarship for the technician)}.
\end{tabularx}


%%%%%%%%%%%%%%%%%%%%%%%%%%%%%%%%%%%%%%%%%%%%%%%%%%%%%%%%%%%%%%%%%%%%%%%%%%%%%%%
\section*{Honors \& Awards}

\TablePad
\begin{tabularx}{\textwidth}{@{}p{0.15\textwidth} p{0.85\textwidth}}
    \Year{2017}  &
    Brazilian Geophysical Society (SBGf) Award for \textbf{Best PhD Thesis}
    of 2015-2017
    \\
    \Year{2016}  &
    Universidade do Estado do Rio de Janeiro, Brazil, School of Geology
    \textbf{Teaching Award} given by the graduating class of 2016
    \\
    \Duration{2011}{2015}  &
    Brazilian Ministry of Education CAPES \textbf{PhD Scholarship} (4 years)
    with Professor Valéria C. F. Barbosa
    \\
    \Year{2011}  &
    SEG Near Surface Geophysics Section \textbf{Student Travel Grant} to
    present at the SEG Annual Meeting, San Antornio, TX, USA
    \\
    \Year{2011}  &
    EAGE \textbf{PACE Student Travel Grant} to present at the 73rd EAGE
    Conference \& Exhibition, Vienna, Austria
    \\
    \Duration{2010}{2011}  &
    Brazilian Ministry of Education CAPES \textbf{Masters Scholarship} (2
    years) with Professor Valéria C. F. Barbosa
    \\
    \Year{2008}  &
    Brazilian Geophysical Society (SBGf) \textbf{Undergraduate Research
    Scholarship} (1 year) with Professor Naomi Ussami
    \\
    \Year{2005}  &
    São Paulo Research Foundation (FAPESP) \textbf{Undergraduate Research
    Scholarship} (1 year) with Professor Manoel S. D'Agrella Filho
\end{tabularx}


%%%%%%%%%%%%%%%%%%%%%%%%%%%%%%%%%%%%%%%%%%%%%%%%%%%%%%%%%%%%%%%%%%%%%%%%%%%%%%%
\section*{Open-source Software}

\TablePad
\begin{tabularx}{\textwidth}{@{}X X X@{}}
    \SoftwareTitle{Fatiando a Terra}
    \newline
    A Python library for geophysical data analysis, modeling, and
    inversion.
    \newline
    \href{https://github.com/fatiando/fatiando}{github.com/fatiando/fatiando}
    &
    \SoftwareTitle{Tesseroids}
    \newline
    Command-line programs for forward modeling of gravitational fields in
    spherical coordinates.
    \newline
    \href{https://github.com/leouieda/tesseroids}{github.com/leouieda/tesseroids}
    &
    \SoftwareTitle{GMT/Python}
    \newline
    A Python interface for the Generic Mapping Tools.
    \newline
    \href{https://github.com/GenericMappingTools/gmt-python}{github.com/GenericMappingTools/\newline gmt-python}
\end{tabularx}


%%%%%%%%%%%%%%%%%%%%%%%%%%%%%%%%%%%%%%%%%%%%%%%%%%%%%%%%%%%%%%%%%%%%%%%%%%%%%%%
\section*{Teaching}

\TablePad
\begin{tabularx}{\textwidth}{@{}X X@{}}
    \TableTitle{Undergraduate} & \TableTitle{Workshops/Short courses}
    \\[0.1cm]
    \begin{tabular}{@{}l l}
        \Duration{2014}{2016}  &
         matesp
        \hspace{10cm}
        \\
        \Duration{2014}{2016}  &
         Geofisica 1
        \\
        \Duration{2014}{2016}  &
         Geofisica 2
        \\
        \Year{2015}  &
         Intro Geology
    \end{tabular}
    &
    \begin{tabular}{@{}l l}
        \Year{2017}  &
        Short course Python UH
        \\
        \Year{2016}  &
        Python USP
        \\
        \Year{2014}  &
        Inversion UnB
        \\
        \Year{2011}  &
        Inversion USP
    \end{tabular}
\end{tabularx}


%%%%%%%%%%%%%%%%%%%%%%%%%%%%%%%%%%%%%%%%%%%%%%%%%%%%%%%%%%%%%%%%%%%%%%%%%%%%%%%
\section*{Publications}

\subsection*{Journal articles}

\TablePad
\begin{tabularx}{\textwidth}{@{}l X}
\Year{2017}  &
    \textbf{Uieda, L}.
    Step-by-step NMO correction,
    \emph{The Leading Edge},
    \DOI{10.1190/tle36020179.1}.
    \PDF{http://www.leouieda.com/papers/nmo-tutorial.html}
    \Code{https://github.com/pinga-lab/nmo-tutorial}
    \\
    ~ &
    \textbf{Uieda, L} and Barbosa, VCF.
    Fast non-linear gravity inversion in spherical coordinates with application
    to the South American Moho,
    \emph{Geophysical Journal International},
    \DOI{10.1093/gji/ggw390}.
    \PDF{http://www.leouieda.com/papers/paper-moho-inversion-tesseroids-2016.html}
    \Code{https://github.com/pinga-lab/paper-moho-inversion-tesseroids}
    \\
\Year{2016}  &
    \textbf{Uieda, L}, Barbosa, VCF and Braitenberg, C.
    Tesseroids: forward modeling gravitational fields in spherical coordinates,
    \emph{Geophysics},
    \DOI{10.1190/geo2015-0204.1}.
    \PDF{http://www.leouieda.com/papers/paper-tesseroids-2016.html}
    \Code{https://github.com/pinga-lab/paper-tesseroids}
    \\
    ~ &
    Carlos, DU, \textbf{Uieda, L}, and Barbosa, VCF.
    How two gravity-gradient inversion methods can be used to reveal different
    geologic features of ore deposit - A case study from the Quadrilátero
    Ferrífero (Brazil),
    \emph{Journal of Applied Geophysics},
    \DOI{10.1016/j.jappgeo.2016.04.011}.
    \PDF{http://www.leouieda.com/papers/paper-quadrilatero2-2016.html}
    \\
\Year{2015}  &
    Oliveira Jr, VC, Sales, DP, Barbosa, VCF, and \textbf{Uieda, L}.
    Estimation of the total magnetization direction of approximately spherical
    bodies,
    \emph{Nonlinear Processes in Geophysics},
    \DOI{10.5194/npg-22-215-2015}.
    \PDF{http://www.leouieda.com/papers/paper-mag-dir-2015.html}
    \Code{https://github.com/pinga-lab/Total-magnetization-of-spherical-bodies}
    \\
\Year{2014}  &
    \textbf{Uieda, L}, Oliveira Jr, VC, and Barbosa, VCF.
    Geophysical tutorial: Euler deconvolution of potential-field data,
    \emph{The Leading Edge},
    \DOI{10.1190/tle33040448.1}.
    \PDF{http://www.leouieda.com/papers/paper-tle-euler-tutorial-2014.html}
    \Code{https://github.com/pinga-lab/paper-tle-euler-tutorial}
    \\
    ~ &
    Carlos, DU, \textbf{Uieda, L}, and Barbosa, VCF.
    Imaging iron ore from the Quadrilátero Ferrífero (Brazil) using geophysical
    inversion and drill hole data,
    \emph{Ore Geology Reviews},
    \DOI{10.1016/j.oregeorev.2014.02.011}.
    \PDF{http://www.leouieda.com/papers/paper-quadrilatero-2014.html}
    \\
\Year{2013}  &
    Melo, FF, Barbosa, VCF, \textbf{Uieda, L}, Oliveira Jr, VC, and Silva, JBC.
    Estimating the nature and the horizontal and vertical positions of 3D
    magnetic sources using Euler deconvolution,
    \emph{Geophysics},
    \DOI{10.1190/geo2012-0515.1}.
    \PDF{http://www.leouieda.com/papers/paper-euler-plateau-2013.html}
    \\
    ~ &
    Oliveira Jr, VC, Barbosa, VCF, and \textbf{Uieda, L}.
    Polynomial equivalent layer,
    \emph{Geophysics},
    \DOI{10.1190/geo2012-0196.1}.
    \PDF{http://www.leouieda.com/papers/paper-polynomial-eqlayer-2013.html}
    \\
\Year{2012}  &
    \textbf{Uieda, L} and Barbosa, VCF.
    Robust 3D gravity gradient inversion by planting anomalous densities,
    \emph{Geophysics},
    \DOI{10.1190/geo2011-0388.1}.
    \PDF{http://www.leouieda.com/papers/paper-planting-anomalous-densities-2012.html}
    \Code{https://github.com/pinga-lab/paper-planting-densities}
\end{tabularx}


\subsection*{Conference proceedings}

\TablePad
\begin{tabularx}{\textwidth}{@{}l X}
%Melo, F. F., Barbosa, V. C. F., Uieda, L., Jr, V. C. O., and Silva, J. B. C. (2014). A Single Euler Solution Per Anomaly. In 76th EAGE Conference and Exhibition 2014. https://doi.org/10.3997/2214-4609.20140891

%Uieda, L., Oliveira Jr, V. C., and Barbosa, V. C. F. (2013). Modeling the Earth with Fatiando a Terra. In S. van der Walt, J. Millman, and K. Huff (Eds.), Proceedings of the 12th Python in Science Conference (pp. 91–98). Retrieved from http://conference.scipy.org/proceedings/scipy2013/uieda.html

%Uieda, L., and Barbosa, V. C. F. (2012). Use of the “shape-of-anomaly” data misfit in 3D inversion by planting anomalous densities (pp. 1–6). Society of Exploration Geophysicists. https://doi.org/10.1190/segam2012-0383.1

%Carlos, D. U., Uieda, L., Li, Y., Barbosa, V. C. F., Braga, M. A., Angeli, G., and Peres, G. (2012). Iron ore interpretation using gravity-gradient inversions in the Carajás, Brazil. In SEG Annual Meeting (pp. 1–5). Society of Exploration Geophysicists. https://doi.org/10.1190/segam2012-0525.1

%Oliveira Jr., V. C., Barbosa, V. C. F., and Uieda, L. (2012). Camada Equivalente Polinomial. In V Simpósio Brasileiro de Geofísica. Retrieved from http://www.earthdoc.org/publication/publicationdetails/?publication=67523

%Uieda, L., Bomfim, E. P., Braitenberg, C., and Molina, E. (2011). Optimal forward calculation method of the Marussi tensor due to a geologic structure at GOCE height. In Proceedings of the 4th International GOCE User Workshop. Retrieved from http://www.lithoflex.org/bib/Munich2011_GOCEU.pdf

%Uieda, L., and Barbosa, V. C. F. (2011c). Robust 3D gravity gradient inversion by planting anomalous densities. In SEG Annual Meeting. https://doi.org/10.1190/1.3628201

%Uieda, L., and Barbosa, V. C. (2011a). 3D gravity inversion by planting anomalous densities. In 12th International Congress of the Brazilian Geophysical Society. Retrieved from http://www.earthdoc.org/publication/publicationdetails/?publication=55052

%Uieda, L., and Barbosa, V. C. F. (2011b). 3D gravity gradient inversion by planting density anomalies. In 73th EAGE Conference and Exhibition incorporating SPE EUROPEC. https://doi.org/10.3997/2214-4609.20149567

%Carlos, D. U., Uieda, L., Barbosa, V. C. F., Braga, M. A., and Gomes, A. A. S. (2011). In-depth imaging of an iron orebody from Quadrilatero Ferrifero using 3D gravity gradient inversion. In SEG Annual Meeting (pp. 902–906). Society of Exploration Geophysicists. https://doi.org/10.1190/1.3628219

\Year{2011}  &
    Carlos, DU, Barbosa, VCF, Uieda, L, and Braga, MA.
    Inversão de Dados de Aerogradiometria Gravimétrica 3D-FTG Aplicada a
    Exploração Mineral na Região do Quadrilátero Ferrífero.
    \emph{12th International Congress of the Brazilian Geophysical Society}.
    \DOI{10.1190/sbgf2011-243}.
\end{tabularx}


%%%%%%%%%%%%%%%%%%%%%%%%%%%%%%%%%%%%%%%%%%%%%%%%%%%%%%%%%%%%%%%%%%%%%%%%%%%%%%%
\section*{Presentations
          \lowercase{\fontsize{11pt}{0}\selectfont (first author only)}}

\TablePad
\begin{tabularx}{\textwidth}{@{}l X}
    \Year{2017}  &
    SBGf Premio tese
    \hspace{0.9\textwidth}
    \\
    \Year{2016}  &
    Paraninfo
\end{tabularx}


%%%%%%%%%%%%%%%%%%%%%%%%%%%%%%%%%%%%%%%%%%%%%%%%%%%%%%%%%%%%%%%%%%%%%%%%%%%%%%%
\section*{Skills}

\TablePad
\begin{tabularx}{\textwidth}[t]{@{}p{0.5\textwidth} p{0.5\textwidth}@{}}
    \TableTitle{Languages} &
    \TableTitle{Programming}
    \\[0.1cm]
    \begin{tabular}[t]{@{}l}
        \Item English: fluent (TOEFL iBT score 115/120)
        \\
        \Item Portuguese: native
        \\
        \Item Spanish: basic
    \end{tabular}
    &
    \begin{tabular}[t]{@{}l}
        \Item Python (main language since 2008)
        \\
        \Item C
        \\
        \Item Bash
        \\
        \Item Version control (git, mercurial, subversion)
        \\
        \Item HTML \& CSS
        \\
        \Item LaTeX
    \end{tabular}
\end{tabularx}


%%%%%%%%%%%%%%%%%%%%%%%%%%%%%%%%%%%%%%%%%%%%%%%%%%%%%%%%%%%%%%%%%%%%%%%%%%%%%%%
\section*{Service}

\TablePad
\begin{tabularx}{\textwidth}[t]{@{}p{0.5\textwidth} p{0.5\textwidth}@{}}
    \TableTitle{Affiliations}
    &
    \TableTitle{Reviewer}
    \\[0.1cm]
    \begin{tabular}[t]{@{}l}
        \Item American Geophysical Union
        \\
        \Item Society of Exploration Geophysicists
        \\
        \Item Geological Society of America
    \end{tabular}
    &
    \begin{tabular}[t]{@{}l}
        \Item Computers \& Geosciences
        \\
        \Item Geophysics
        \\
        \Item Central European Journal of Geosciences
        \\
        \Item Pure and Applied Geophysics
        \\
        \Item Journal of Applied Geophysics
        \\
        \Item Geophysical Prospecting
        \\
        \Item Geophysical Journal International
        \\
        \Item Journal of Geodesy
    \end{tabular}
\end{tabularx}

\end{document}
